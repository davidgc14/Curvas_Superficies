\documentclass[fleqn]{article}

\usepackage{mathexam}
\usepackage{amsmath}
\usepackage{amsfonts}
\usepackage{graphicx}
\usepackage{systeme}
\usepackage{microtype}
\usepackage{multirow}
\usepackage{pgfplots}
\usepackage{listings}
\usepackage{tikz}
\usepackage{dsfont} %Numeros reales, naturales...

%\graphicspath{{images/}}
\newcommand*{\qed}{\hfill\ensuremath{\square}}

%Estructura de ecuaciones
\setlength{\textwidth}{15cm} \setlength{\oddsidemargin}{5mm}
\setlength{\textheight}{23cm} \setlength{\topmargin}{-1cm}



\author{David García Curbelo}
\title{Curvas y Superficies}
\date{Grado en Matemáticas, Grupo A}

\pagestyle{empty}



\def\R{\mathds{R}}
\def\sup{$^2$}

\begin{document}
    
    \maketitle
    \setcounter{page}{1}
    \pagestyle{plain}

    {\bf{Ejercicio 3:}} Se considera la espiral logarítmica $\alpha : \R \longrightarrow \R$ definida por
    $$\alpha (t) = e^t \left(\cos t, \sin t\right)$$

    \begin{itemize}
        \item Calcula su función longitud de arco.
        \item Calcula su función curvatura.
        \item Comprueba que los vectores de posición y tangente en cada punto forman un ángulo constante.\\ \\
    \end{itemize}

    {\bf{Solución:}} Para calcular su función longitud de arco consideramos primero la parametrización por el
    arco de la curva $\alpha (t)$. Sabemos que la parametrización viene dada por 
    $s(t) = L _{t_0}^t (\alpha) = \int _{t_0}^t |\alpha '(v)| dv$

    $$\alpha '(t) =\left(e^t \cos t - e^t \sin t, e^t\sin t + e^t\cos t\right)$$
    $$|\alpha '(t)|=e^t \sqrt{2}$$
    Y por lo tanto, tomando $t_0=0$ obtenemos 
    $$s(t) = \int _{0}^t |e^v \sqrt{2}| dv = \sqrt{2}(e^t-1)$$
    Entonces $s=\sqrt{2}(e^t-1)$ y por lo tanto $t=\ln \left(\frac{\sqrt{2}}{2}s+1\right)$, y obtenemos que la 
    función 
    $$\beta (s)=\left(\frac{\sqrt{2}}{2}s+1\right) \left(\cos \left(\ln \left(\frac{\sqrt{2}}{2}s+1\right)\right), 
    \sin \left(\ln \left(\frac{\sqrt{2}}{2}s+1\right)\right)\right)$$
    es la reparametrización por el arco de la curva $\alpha (t)$, y por lo tanto su función longitud de arco. \\ \\

    Calculamos ahora su función curvatura. La curvatura de nuestra curva p.p.a $\beta (s)$ viene dada por las 
    ecuaciones de Frenet de la curva:
    \begin{equation*}
        \left.
            \begin{aligned}
                T'(s)=k(s)N(s)\\
                N'(s)=-k(s)T(s)
            \end{aligned}
        \right\}
    \end{equation*}
    Para el que sabemos que $T(s)$ y $N(s)$ son los vectores tangente y normal de la curva respectivamente, y $k(s)$
    la función curvatura. Como el vector tangente viene dado por la derivada de la curva, obtenemos $T(s)=\beta '(s)$
    y por tanto $T'(s)=\beta ''(s)$. Como el vector normal es perpendicular al vector tangente, podemos obtener dicho 
    vector mediante $N(s)=J(T(s))=J((x,y))=(-y,x)$. Transcribimos los datos que hemos obtenido:
    {\small{$$T(s)= \frac{\sqrt{2}}{2} \left(\cos\left\{\ln \left(\frac{\sqrt{2}}{2}s +1\right)\right\} - \sin\left\{\ln \left(\frac{\sqrt{2}}{2}s +1\right)\right\},
    \sin\left\{\ln \left(\frac{\sqrt{2}}{2}s +1\right)\right\} + \cos\left\{\ln \left(\frac{\sqrt{2}}{2}s +1\right)\right\} \right)$$}

    \small{$$T'(s)= \frac{1}{s\sqrt{2}+2} \left(-\sin\left\{\ln \left(\frac{\sqrt{2}}{2}s +1\right)\right\} - \cos\left\{\ln \left(\frac{\sqrt{2}}{2}s +1\right)\right\},
    \cos\left\{\ln \left(\frac{\sqrt{2}}{2}s +1\right)\right\} - \sin\left\{\ln \left(\frac{\sqrt{2}}{2}s +1\right)\right\} \right) $$}
}
    Sabemos también que $N(s)=J(T(s))$, y tomando la igualdad $T'(s)=k(s)N(s)$, reescribiendo queda $\frac{1}{k(s)}T'(s)=N(s)$ donde obtenemos
    $$\frac{1}{k(s)}=\left(s\sqrt{2}+2\right)\frac{\sqrt{2}}{2}=s+\sqrt{2}$$
    $$k(s)=\frac{1}{s+\sqrt{2}}$$
    Con la que obtenemos la función curvatura que buscábamos.\\ \\

    Para el tercer apartado tomamos el vector posición dado por $ \alpha=e^t \left(\cos t, \sin t\right)$ y su vector tangente  en función de $t$, dado por
    $T(t)=\frac{\sqrt{2}}{2} \left(\cos t - \sin t, \sin t + \cos t\right)$. Calculamos el ángulo que forman ambos mediante 
    $\cos \theta = \frac{\alpha(t) \cdot T(t)}{|\alpha(t)| \cdot |T(t)|}$
    $$\cos \theta = \frac{\frac{\sqrt{2}}{2}e^t(\cos^2t + \sin^2t)}{\sqrt{e^{2t}(\cos^2t + \sin^2t)\frac{1}{2}(\cos^2t + \sin^2t)}}$$
    $$\cos \theta = \frac{\frac{\sqrt{2}}{2}e^t}{\sqrt{\frac{1}{2}e^{2t}}}$$
    $$\theta=\frac{\pi}{2}$$
    Luego el angulo es constante, ya que no depende del tiempo.\qed 
   
    \newpage

    {\bf{Ejercicio 5:}} Sea $\alpha : \R \rightarrow \R ^3$ una curva regular parametrizada como 
    $$\alpha(t)=\left(1 + \cos t, \sin t, 2(1-\cos t)\right)$$
    ¿Es una curva plana? Si lo es, calcula el plano que la contiene.\\ \\

    {\bf{Solución:}} Para ver si es plana bastará con ver si su torsión o su curvatura son nulas en todo su dominio de definición con respecto
    a un determinado plano (el plano osculador). Para ello calculamos sus vectores tangente, normal y binormal
    $$T(t)=\alpha '(t)=\left(-\sin t, \cos t, 2\sin t\right)$$
    $$T'(t)=\left(-\cos t,- \sin t, 2\cos t\right)$$
    $$N(t)=\frac{T'(t)}{|T'(t)|}=\frac{\left(-\cos t,- \sin t, 2\cos t\right)}{\sqrt{1+4\cos^2 t}}$$
    Donde vemos que la curvatura $k(t)$ viene dada por
    $$k(t)=|T'(t)|=\sqrt{1+4\cos^2 t}$$
    Y el binormal por el producto vectorial de los vectores tangente y normal
    $$B(t)=T(t)\times N(t)=\frac{1}{\sqrt{1+4\cos^2 t}} \left|
    \begin{matrix}
        \hat{x} & \hat{y} & \hat{z} \\
        -\sin t & \cos t & 2\sin t \\
        -\cos t & -\sin t & 2\cos t \\
    \end{matrix} \right|
    = \frac{1}{\sqrt{1+4\cos^2 t}} (\left(2,0,1\right)) $$
    
    Como sabemos, el binormal es el vector normal del plano osculador formado por los vectores tangencial y normal. Veamos si este plano tiene
    alguna dependencia en $t$. Si dicho plano no tiene dependencia en $t$ implicará que los vectores tangente y normal se mueven en un mismo plano
    al recorrer la curva $\alpha(t)$, y que por tanto la torsión sera nula (y por tanto la curva será plana). Tomamos un punto cualquiera de la curva
    $(x_0, y_0, z_0)$. La ecuación de dicho plano viene dada por
    $$B_1(t)(x-x_0) + B_2(t)(y-y_0) + B_3(t)(z-z_0)=0$$
    $$\frac{2}{\sqrt{1+4\cos^2 t}}(x-x_0) + \frac{0}{\sqrt{1+4\cos^2 t}}(y-y_0) + \frac{1}{\sqrt{1+4\cos^2 t}}(z-z_0)=0$$
    $$\frac{2}{\sqrt{1+4\cos^2 t}}(x-x_0) + \frac{1}{\sqrt{1+4\cos^2 t}}(z-z_0)=0$$
    $$2(x-x_0) + 1(z-z_0)=0$$

    Vemos que el plano no tiene dependencia en $t$, por lo tanto su torsión es nula y podemos afirmar que se trata de una curva plana.
    Además, tomando un punto cualquiera de la curva, por ejemplo $\alpha(0)=(2,0,0)=(x_0, y_0, z_0)$ y sustituyéndolo en el plano antes calculado

    $$2x+z=4$$
    obtenemos el plano osculador, y por tanto el plano que contiene a la curva plana.


\end{document}