\documentclass[fleqn]{article}

%\pgfplotsset{compat=1.17}

\usepackage{mathexam}
\usepackage{amsmath}
\usepackage{amsfonts}
\usepackage{graphicx}
\usepackage{systeme}
\usepackage{microtype}
\usepackage{multirow}
\usepackage{pgfplots}
\usepackage{listings}
\usepackage{tikz}
\usepackage{dsfont} %Numeros reales, naturales...

%\graphicspath{{images/}}
\newcommand*{\QED}{\hfill\ensuremath{\square}}

%Estructura de ecuaciones
\setlength{\textwidth}{15cm} \setlength{\oddsidemargin}{5mm}
\setlength{\textheight}{23cm} \setlength{\topmargin}{-1cm}



\author{David García Curbelo}
\title{Curvas y Superficies, Prueba 3 Opción A}
\date{Grado en Matemáticas, Grupo A}

\pagestyle{empty}



\def\R{\mathds{R}}
\def\sup{$^2$}

\begin{document}
    \maketitle
    \setcounter{page}{1}
    \pagestyle{plain}

    \textbf{Ejercicio 1. } \\

    No es relevante que dos superficies sean difeomorfas para que tengan la misma imagen esférica. Sabemos que ambas superficies (el cilindro elíptico y el catenoide) 
    son superficies difeomorfas, y además ambas son orientables. Por ser orientables podemos cosniderar una orientación para cada una de las superficies, y por tanto una aplicación 
    de Gauss para cada una de ellas, dada por 
    \begin{equation*}
        N_1: S_1 \longrightarrow \mathds{S}^2\\
        N_2: S_2 \longrightarrow \mathds{S}^2
    \end{equation*}
    siendo $S_1$ y $S_2$ el cilindro y el catenoide, respectivamente. Calculemos primero la imagen esférica del cilindro. para ello consideramos el conjunto
    $$C_{\alpha} = \{\alpha(t) + v\vec{e}_3 \thinspace : \thinspace t\in I, \thinspace v\in \R\}$$
    Donde $\alpha(t)$ viene dado por la elipse $\alpha(t) = (a\cos(t), b\sin(t), 0), \thinspace a,b\in \R$. Consideremos por tanto la siguiente aplicación, en la que nos basaremos para 
    la construcción de $N_1$.
    \begin{equation*}
        \begin{aligned}
            & F:I\times \R \rightarrow C_{\alpha} \subset \R^3 \\
            &F(t,v) = (a\cos(t), b\sin(t), v)
        \end{aligned}
    \end{equation*}
    
    Por tanto obtenemos la fórmula de Gauss:
    $$N_1(F(t,v)) = \frac{F_t \times F_v}{\left|F_t \times F_v\right|} = \frac{1}{\sqrt{a^2\sin^2(t) + b^2\cos^2(t)}} (b\cos(t), a\sin(t), 0)$$
    Donde podemos observar que dicha aplicación de Gauss lleva cualquier cilindro elíptico a la circunferencia obtenida de intersecar la esfera $\mathds{S}^2$ con el plano $z=0$. Calculemos
    ahora la aplicación de Gauss para el catenoide. Sabemos que se trata de una superficie de revolución, al rotar sobre (en este caso) el eje $z$ una catenaria de la forma
    $\alpha(t) = (a\cosh(t), 0, bt), \thinspace b,a\in \R-\{0\}$. Por ello, obtengamos la imagen esférica del campo normal unitario de nuestra catenaria, que tendrá función de curva generatriz,
    el cual rotaremos posteriormente para obtener así la imagen esférica de toda la superficie. Para ello consideremos su referencia de Frenet, la cual viene dada por
    $$N(t) = \left(\frac{\sinh(t)}{\cosh(t)}, 0, \frac{1}{\cosh(t)} \right)$$
    El cual se trata de una semicircunferencia sin los extremos, y que por tanto al rotarla obtenemos
    que la imagen esférica del catenoide es toda la esfera $\mathds{S}^2$ sin las antípodas que pasan por el eje $z$, es decir, $\mathds{S}^2 - \{(0,0,1), (0,0,-1)\}$, y por tanto podemos ver que,
    a pesar de ser dos superficies difeomorfas, poseen imágenes esféricas diferentes.\\ \\

    \textbf{Ejercicio 2. } \\

    El enunciado mencionado es claramente cierto. 

    \textbf{Ejercicio 3. } \\

    Para poder comprobar la veracidad o falsedad del enunciado, necesitamos saber la función curvatura de Gauss en toda la superficie, para poder así clasificar sus puntos. Sabemos que el 
    toro elíptico es una superficie de revolución resultado de rotar con respecto al eje $z$ la curva $\alpha(t) = (R + a\cos(t), 0, b\sin(t))$, donde $R>a, \thinspace R,a,b > 0$. Sabiendo esto,
    podemos definir la matriz del endomorfismo de Weingarten como sigue 
    $$A=
    \begin{pmatrix}
        k(t) & 0 \\
        0 & \frac{\alpha_3'(t)}{\alpha_1(t)}
    \end{pmatrix}
    $$
    donde $k(t)$ es la curvatura de la curva $\alpha(t)$, la cual sabemos que viene dada por $k(t)=\frac{ab}{(a^2\sin^2(t) + b^2 \cos^2(t))^{\frac{3}{2}}}$, que además es positiva
    para todo $t\in \R$. Por ello, como sabemos que las función de curvatura de Gauss viene dada por el producto de los valores propios de dicha matriz, vemos que
    $$K = \frac{ab}{(a^2\sin^2(t) + b^2 \cos^2(t))^{\frac{3}{2}}} \frac{\alpha_3'(t)}{\alpha_1(t)} = \frac{ab^2 \cos(t)}{(R + a\cos(t))(a^2\sin^2(t) + b^2 \cos^2(t))^{\frac{3}{2}}}$$
    Estudiando el signo de dicha igualdad, observamos que cuando $t=\frac{\pi}{2}$ ó $t=-\frac{\pi}{2}$, tenemos que $K=0$, por lo que los paralelos generados al girar estos dos puntos
    de la curva $\alpha(t)$ están compuestos por puntos parabólicos. Por ello podemos concluir que un toro elíptico NO tiene todos sus puntos elípticos, ya que contiene otros tipos de puntos. \\ \\



    \textbf{Ejercicio 5. } \\

    Es verdadero. Basta con ver que un paralelo de una superficie de revolución es el resultado de girar un cierto punto $\alpha(t_0)$ de la curva generatriz de la superficie con respecto
    al eje de giro. Por ello, al clasificar un cierto punto de dicha curva generatriz, que es clasificado mediante la función curvatura de Gauss, vemos que $K$ no depende del ángulo rotado,
    y por tanto la clasificación del punto se mantiene a lo largo del paralelo. Como ejemplo podemos considerar el toro elíptico calculado en el \textit{Ejercicio 3}, en el cual vimos que 
    cuando la curva generatriz tomaba el valor $\frac{\pi}{2}$ (y por tanto considerábamos dicho punto $\alpha(\frac{\pi}{2})$), vimos que el valor de la curvatura de Gauss era nula, y concluíamos
    que, tanto dicho punto como todo su paralelo, estaban constituidos por puntos parabólicos.

\end{document}