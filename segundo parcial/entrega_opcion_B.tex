\documentclass[fleqn]{article}

%\pgfplotsset{compat=1.17}

\usepackage{mathexam}
\usepackage{amsmath}
\usepackage{amssymb}
\usepackage{amsfonts}
\usepackage{graphicx}
\usepackage{systeme}
\usepackage{microtype}
\usepackage{multirow}
\usepackage{pgfplots}
\usepackage{listings}
\usepackage{tikz}
\usepackage{dsfont} %Numeros reales, naturales...
\usepackage{cancel}
\usepackage{verbatim} %Comentario de parrafo

%\graphicspath{{images/}}
\newcommand*{\QED}{\hfill\ensuremath{\square}}

%Estructura de ecuaciones
\setlength{\textwidth}{15cm} \setlength{\oddsidemargin}{5mm}
\setlength{\textheight}{23cm} \setlength{\topmargin}{-1cm}



\title{Curvas y Superficies}
\author{David García Curbelo}
\date{Grado en Matemáticas, Grupo A}

\pagestyle{empty}


\def\R{\mathds{R}}
\def\Z{\mathds{Z}}
\def\N{\mathds{N}}
\def\sup{$^2$}


\begin{document}
        \maketitle
        \setcounter{page}{1}
        \pagestyle{plain}

        \textbf{Ejercicio 1. } 
        
        Consideremos una superficie cualquiera $S\in \R^3$ que sea regular y compacta. Para que dicha superficie sea grafo de 
        una función diferenciable, debe de existir una parametrización 
        $$X: \thinspace U\subset \R^2 \longrightarrow \R^3$$
        con $U$ un abierto del plano, tal que se verifique $X(u,v) = (u, v, h(u,v))$, $\forall (u,v) \in U$, con $h(u,v)$ una función
        diferenciable (con la que buscamos $S = X(u,v)$). Como $X$ es una parametrización, se trata de un difeomorfismo, y por lo tanto podemos afirmar
        que $S$ es homeomorfa a un abierto $U$ del plano. Además, como la compacidad es una invariante topológica, y sabemos que $S$ es compacta, entonces tenemos que
        $U$ tiene que ser compacto. Sin embargo, un abierto nunca puede ser compacto en $\R^n$ debido a la caracterización de la compacidad en $\R^n$, 
        por lo que hemos llegado a una contradicción. Por ello concluimos que una superficie regular compacta no puede ser nunca el grafo de una 
        función diferenciable.\\ \\



        \textbf{Ejercicio 2. } 
        
        Para demostrar que ambas superficies son difeomorfas, usaremos parametrizaciones de Monge paracomprobar que ambas 
        superficies son difeomorfas a un abierto de $\R^2$.

        La catenaria viene dada por una curva 
        $$\alpha : \R \longrightarrow \R^3, \quad \alpha(t)=(t, \cosh (t), 0)$$
        Y por lo tanto, el cilindro sobre la catenaria se representa mediante el siguiente conjunto
        $$S_1 = \{\alpha(t) + v\cdot (0,0,1) \thinspace : \thinspace t,v \in \R\}$$
        Por ello, vamos a continuación a proceder a la construcción de una parametrización $X_1$ que cubra a toda la superficie $S_1$
        $$X_1: \thinspace \R^2 \longrightarrow \R^3, \quad X_1(u,v) = (u, \cosh (u), v)$$
        Vemos que dicha parametrización es una función continua, y además inyectiva, por lo que podemos afirmar que se trata de un homemomorfismo.
        A continuación calcularemos el rango del Jacobiando para ver que se trata de un difeomorfismo:
        $$J(X_1)= \left|
                \begin{matrix}
                        1 & \sinh(u) & 0 \\
                        0 & 0 & 1
                \end{matrix}
        \right|$$
        Del cual podemos observar fácilmente que se trata de una matriz de rango 2 (estudiando el valor del determinante formado por la primera y 
        tercera columna). Como $X_1$ es una parametrización que cubre toda la superficie, obtenemos 
        que $X_1$ es un difeomorfismo, y por ello $S_1$ es difeomorfa al plano $\R^2$.

        A continuación realizamos el mismo estudio pero para el helicoide, con ecuación 
        $$S_2 = \{(x,y,z) \in \R^3 \thinspace : \thinspace y/x = \tan(z)\}$$
        Y definimos para el helicoide la siguiente parametrización
        $$X_2 = \thinspace \R^2 \longrightarrow \R^3, \quad X_1(u,v) = (u \cos(v), u \sin (v), v)$$
        Donde es fácil ver que $\frac{u \sin(v)}{u \cos (v)} = \tan (v)$, y por tanto obtenemos que $S_2 = X_2$, con lo que podemos afirmar que la 
        parametrización recubre toda la superficie $S_2$. Para ver ahora que se trata efectivamente de un difeomorfismo, basta con ver (como en el 
        apartado anterior) que sea un homeomorfismo y que el rango de su matriz Jacobiana asociada tenga rango 2. Es fácil ver que se trata de un
        homeomorfismo, ya que se trata de una aplicación continua e inyectiva. Veamos ahora el rango de su martiz Jacobiana
        $$J(X_1)= \left|
                \begin{matrix}
                        \cos(v) & \sin(v) & 0 \\
                        -u\sin(v) & u\cos(v) & 1
                \end{matrix}
        \right|$$
        Del cual podemos observar que, estudiando el determinante asociado a laas dos primeras columnas vemos que es no nulo (con $u\neq 0$), 
        y que por tanto el Jacobiano tiene rango 2. Tenemos por ello que $X_2$ es un difeomorfismo que además que cubre toda la superficie $S_2$, y por tanto
        $S_2$ es difeomorfa al plano $R_2$. 

        Por tanto, por la porpiedad de transitividad, como ambas superficies $S_1$ y $S_2$ son difeomorfas a $\R^2$, concluimos que ambas superficies
        son difeomorfas.\\ \\

        \textbf{Ejercicio 3. }
        
        Para ver que las dos superficies son difeomorfas basta con encontrar una aplicación (difeomorfismo) que relacione la esfera, que tiene por 
        ecuación 
        $$S_1 = \{(x,y,z)\in \R^3\thinspace : \thinspace (\frac{x-2}{247})^2 + (\frac{y+1}{247})^2 + (\frac{z-1}{247})^2 = 1\}$$
        con el elipsoide, cuya ecuación viene dada por 
        $$S_2 = \{(x,y,z)\in \R^3\thinspace : \thinspace x^2 + (\frac{y}{6})^2 + (\frac{z}{8})^2 = 1\}.$$

        Consideremos por ello una aplicación 

                $$F: S_1\subset \R^3 \longrightarrow S_2\subset \R^3 , \quad \quad \quad
                F(x,y,z)=\left(247x + 2, \thinspace y\frac{247}{6}-1, \thinspace z\frac{247}{8} + 1\right)$$

        Comprobemos que efectivamente se trata de un difeomorfismo. Para ello debemos comprobar
        que se trate de una aplicación biyectiva, y que además sea diferenciable con inversa diferenciable  (y por tanto ambas continuas). Para ello
        bastará con ver que tanto $F$ como $F^{-1}$ son funciones diferenciables e inyectivas. 

        Definimos por ello la función inversa, la cual viene definida como

                $$F^{-1}: S_2\subset \R^3 \longrightarrow S_1\subset \R^3 , \quad \quad \quad
                F^{-1}(x,y,z)=\left(\frac{x-2}{247}, \thinspace \frac{6}{247}(y+1), \thinspace \frac{8}{247}(z - 1)\right)$$
        Vemos claramente que ambas funciones son continuas y de clase $\mathcal{C}^{\infty}$ (particularmente $\mathcal{C}^{1}$), por lo que sólo nos falta 
        comprobar la inyectividad. Para ello veamos si se cumple $F(x,y,z)=F(x',y',z') \Rightarrow (x,y,z)=(x',y',z')$
        $$F(x,y,z)=F(x',y',z') \Rightarrow \left(247x + 2, \thinspace y\frac{247}{6}-1, \thinspace z\frac{247}{8} + 1\right)=
        \left(247x' + 2, \thinspace y'\frac{247}{6}-1, \thinspace z'\frac{247}{8} + 1\right)$$
        $$\Rightarrow (x,y,z)=(x',y',z')$$

        Por lo tanto vemos que es inyectiva. Realizamos el mismo procedimiento para la función inversa $F^{-1}$ y comprobamos su inyectividad
        $$F^{-1}(x,y,z)=F^{-1}(x',y',z') \Rightarrow \left(\frac{x-2}{247}, \thinspace y\frac{6}{247}(y+1), \thinspace \frac{8}{247}(z - 1)\right) = 
        \left(\frac{x'-2}{247}, \thinspace y'\frac{6}{247}(y+1), \thinspace \frac{8}{247}(z' - 1)\right)$$
        $$\Rightarrow (x,y,z)=(x',y',z')$$

        Por lo tanto hemos demostrado que se trata de un difeomorfismo, y por tanto la elipse y la esfera dada son dos superficies difeomorfas.\\ \\



        \textbf{Ejercicio 4. } 
        
        Definimos una función $F$ tal que 
        $$F:\R^3 \rightarrow \R, \quad F(x,y,z)=xyz$$
        la cual es continua, y además diferenciable, donde el gradiente de la función viene dado por 
        $$\nabla F(x,y,z) = (yz,xz,xy).$$

        Buscamos un $a \in F(U)$ con $U$ abierto tal que $a$ sea un valor regular. Para ver que se trata de un valor regular de $F$, ha de verificarse
        que para todo $p \in F^{-1}(a)$, el gradiente de $F$ evaluado en $p$ no se anula. Tomemos por 
        tanto $a=1$, con el que vemos que $F^{-1}(1)=\{(x,y,z) \in \R^3 \thinspace : \thinspace xyz=1\}$. Podemos ver que, en todos los puntos del conjunto,
        ninguna de las componentes de dichos puntos pueden valer cero, ya que no cumpliría la condición $xyz = 1$. Por ello podemos ver que el gradiente evaluado en cualquier
        $p \in F^{-1}(a)$ viene dado por $\nabla F(p_x,p_y,p_z)$ el cual, al cumplirse $p_x,p_y,p_z \neq 0$, tenemos que 
        $$\nabla F(x,y,z)\neq 0, \quad \forall (x,y,z) \in F^{-1}(1)$$
        y por lo tanto podemos afirmar que $a=1$ se trata de un valor regular.
        Por ello, tenemos que el conjunto $F^{-1}(1)=\{(x,y,z) \in \R^3 \thinspace : \thinspace xyz=1\}$ es una superficie regular, como queríamos probar.\\ \\


        \textbf{Ejercicio 5. } 
        
        Si tomamos la definición de cilindro, vemos que se trata de una curva plana $\alpha$ definida en un plano, desplazada por un 
        vector $\vec{v}$ no perteneciente al plano, construyendo así el cilindro. Consideremos por tanto cualquier punto $p$ perteneciente a la curva $\alpha$, que al haber sido 
        trasladado por el vector, hemos obtenido una recta, la cual es la conocida por generatriz. Como podemos ver, la generatriz de cualquier cilindro
        es una recta (por la definición de construcción del cilindro) y por ello, al estudiar el vector normal en un punto cualquiera $p$ de la curva y ser desplazado
        dicho punto por $\vec{v}$, vemos que el vector normal se mantendrá constante a lo largo de toda la recta generada (la generatriz). Si estudiamos 
        ahora el plano tangente en cada punto de la generatriz, al tener en cada punto el mismo vector normal, también se mantendrá constante el plano
        tangente (la generatriz se encuentra contenida en un mismo plano, el plano tangente al cilindro), y por tanto podemos afirmar que los planos tangentes 
        en todos los puntos a lo largo de la generatriz coinciden, como queríamos probar.\\ \\


        \textbf{Ejercicio 6. } 
        
        Para el cálculo del plano tangente a dicha superficie, necesitaremos un punto en el que estudiaremos dicho plano
        (el punto dado $p_0=(1,1,1)$) y un vector perpendicular a la superficie, que calcularemos a continuación (y que será posteriormente nuestro 
        vector normal del plano tangente). Comprobemos que dicho punto $p_0$ pertenece a la superficie a estudiar, sustituyendo en la condición dada
        en el conjunto a estudiar
        $$xy + xz + yz = 3 \quad \Rightarrow \quad 1\cdot 1 + 1\cdot 1 + 1\cdot 1 = 3 \quad \Rightarrow \quad 3=3$$
        Vemos por ello que el punto pertenece al conjunto dado, y por ello podemos proceder a calcular el plano tangente a dicho punto. Definimos 
        una función 
        $$F:R^3 \longrightarrow \R, \quad F(x,y,z)= xy + xz + yz - 3$$
        Sabemos que nuestro vector gradiente evaluado en el punto $p_0$ nos dará el vector perpendicular a la superficie en dicho punto. Por ello 
        calculamos su vector gradiente, el cual viene dado por 
        $$\vec{\nabla} F(x,y,z)= (y+z,x+z,x+y)$$
        Y sustituimos el punto para obtener el vector antes mencionado
        $$\vec{\nabla} F(1,1,1)= (2,2,2) = \vec{n}$$
        que junto con el punto $p_0$ podemos obtener fácilmente la ecuación de la recta $L=(1,1,1) + \langle 2,2,2\rangle$, o de forma simplificada, 
        vemos que viene dada por el origen y el vector $\langle 1,1,1 \rangle$, tal que $L=\langle 1,1,1 \rangle$. Calculamos ahora el plano tangente,
        el cual sabemos que viene dado por el plano perpendicular a la recta $L$ en el punto $(1,1,1)$, y vemos fácilmente que su ecuación es
        $$x+y+z=D$$
        para un cierto $D$ tal que $p_0$ pertenezca al plano, por lo tanto vemos que el plano tangente viene dado por el conjunto
        $$\{(x,y,z)\in \R \thinspace : \thinspace x+y+z=3\}$$

\end{document}